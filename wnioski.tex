\chapter{Wnioski}

W tym dokumencie wykorzystałem klasę \verb|report| zamiast \verb|book|.
Główne powody tego wyboru są następujące:

\begin{itemize}
    \item \textbf{Zakres pracy} – dokument ma kilka rozdziałów, ale nie jest pełnowymiarową książką. 
          Klasa \verb|report| jest przeznaczona właśnie dla raportów, prac zaliczeniowych i projektów.
    \item \textbf{Prostsza struktura} – \verb|report| nie wymaga części typu \textit{front matter}, 
          \textit{main matter}, \textit{back matter}, które są typowe dla \verb|book|.
    \item \textbf{Wystarczające możliwości} – klasa \verb|report| obsługuje rozdziały (\verb|\chapter|),
          sekcje i podsekcje, co w zupełności wystarcza do realizacji wymagań zadania.\\ \\
\end{itemize}


\textbf{Link do repozytorium:}
\url{https://github.com/ptest2901/Zadanie-Latex}
