\chapter{Zastosowanie figur}


\cite{cytat1}Figury geometryczne są wszędzie wokół nas. W architekturze budynki
często opierają się na prostokątach i walcach, a w sztuce i
projektowaniu wykorzystuje się symetrię oraz proporcje geometryczne. W
informatyce geometria stanowi podstawę grafiki komputerowej, modelowania
3D i projektowania gier.\\
\cite{cytat3}Również w przyrodzie można zauważyć geometryczne wzory --- struktura
plastra miodu przypomina sześciokąty, a planety mają kształty zbliżone
do kul.\\



\textbf{Przykłady z życia}

\begin{enumerate}
\def\labelenumi{\arabic{enumi}.}
\item
  \textbf{Sześcian}

  \begin{enumerate}
  \def\labelenumii{\arabic{enumii}.}
  \item
    Kostka do gry
  \item
    pudełko w kształcie sześcianu
  \end{enumerate}
\item
  \textbf{Prostopadłościan}

  \begin{enumerate}
  \def\labelenumii{\arabic{enumii}.}
  \item
    Książka
  \item
    Cegła
  \end{enumerate}
\item
  \textbf{Kula}

  \begin{enumerate}
  \def\labelenumii{\arabic{enumii}.}
  \item
    Piłka
  \item
    Globus
  \end{enumerate}
\item
  \textbf{Ostrosłup}

  \begin{enumerate}
  \def\labelenumii{\arabic{enumii}.}
  \item
    Piramida egipska
  \item
    Stożek drogowy
  \end{enumerate}
\item
  \textbf{Stożek}

  \begin{enumerate}
  \def\labelenumii{\arabic{enumii}.}
  \item
    Rożek do lodów
  \item
    czapka urodzinowa
  \end{enumerate}

\end{enumerate}
