\documentclass[12pt]{report}
\usepackage{graphicx}
\usepackage{float}
\usepackage{titlesec}
\usepackage{hyperref}
\usepackage{fancyhdr}
\setlength{\headheight}{15.2466pt}
\usepackage[utf8]{inputenc}
\usepackage{polski}

\usepackage[right=3cm,
            left=3cm,
            top=2cm,
            bottom=2cm]{geometry}


\title{\Huge{Figury geometryczne}}
\author{\Large{Piotr Śręba}}
\date{8 Grudnia 2025}

\fancypagestyle{plain}{%
  \fancyhf{}
  \fancyhead[L]{\leftmark}
  \fancyhead[R]{Piotr Śręba}
}

\begin{document}


\begin{titlepage}

\maketitle

\end{titlepage}


\addcontentsline{toc}{chapter}{Streszczenie}
\begin{abstract}
Geometria jest jedną z najstarszych dziedzin matematyki. Zajmuje się badaniem kształtów, wymiarów i wzajemnych relacji między punktami, liniami i powierzchniami.\\ Figury geometryczne to podstawowe obiekty, z których zbudowany jest świat przestrzeni. Dzięki nim możemy opisywać zjawiska fizyczne, tworzyć projekty architektoniczne, a nawet analizować dane w naukach przyrodniczych.\\
\end{abstract}

\tableofcontents

\chapter{Figury płaskie}

Figury płaskie to takie, mają tylko dwa wymiary -- długość i szerokość.\\

Do najważniejszych należą:

\begin{itemize}
\item
  \textbf{Trójkąt} -- figura o trzech bokach i trzech kątach; suma jego
  kątów zawsze wynosi 180°.
\item
  \textbf{Kwadrat} -- ma cztery równe boki i cztery kąty proste.
\item
  \textbf{Prostokąt} -- przeciwległe boki są równe, a wszystkie kąty
  proste.
\item
  \textbf{Koło} -- zbiór wszystkich punktów na płaszczyźnie w tej samej
  odległości od środka.\\
\end{itemize}

\includegraphics[scale=1]{Obraz1}\\

\textbf{Przykłady Figur płaskich}

\begin{enumerate}
\def\labelenumi{\arabic{enumi}.}
\item
  Kwadrat
\item
  Trójkąt
\item
  Koło\\
\end{enumerate}

\section{Podział figur płaskich}

\cite{cytat5}Figury płaskie można podzielić na dwie główne grupy: wielokąty oraz figury o krzywoliniowych 
brzegach. Wielokąty, takie jak trójkąty, czworokąty czy pięciokąty, są zbudowane z odcinków 
połączonych wierzchołkami. Figury krzywoliniowe, do których należy koło czy elipsa, mają brzegi 
złożone z linii zakrzywionych. Taki podział pozwala łatwiej porządkować figury i analizować ich 
właściwości.

\section{Elementy charakterystyczne figur płaskich}

Każda figura płaska posiada elementy, które pomagają ją opisać. W przypadku wielokątów są to boki, 
kąty oraz wierzchołki. Długości boków i miary kątów decydują o kształcie i typie figury. W figurach 
okrągłych głównymi elementami są promień, średnica oraz obwód. Dzięki tym elementom można określić 
wielkość figury, obliczyć jej pole oraz porównywać ją z innymi kształtami.






\chapter{Figury przestrzenne}

Figury przestrzenne (bryły) mają trzy wymiary: długość, szerokość i
wysokość. Ich objętość i pole powierzchni są kluczowe w wielu
zastosowaniach. Do najczęściej spotykanych należą:
\\

\begin{table}[H]
\begin{tabular}{| c | c | p{5cm} |}
\hline
\textbf{Numer} & \textbf{Figura} & \textbf{Opis} \\
\hline \hline
\textbf{1.} & \textbf{Sześcian} & wszystkie ściany są kwadratami, a
krawędzie mają równą długość. \\
\hline
\textbf{2.} & \textbf{Prostopadłościan} & ma sześć ścian
prostokątnych. \\
\hline
\textbf{3.} & \textbf{Kula} & zbiór wszystkich punktów w przestrzeni
równo oddalonych od środka. \\
\hline
\textbf{4.} & \textbf{Stożek i walec} & mają podstawy w kształcie koła,
różnią się jednak sposobem łączenia z wierzchołkiem lub drugą
podstawą. \\
\hline
\textbf{5.} & \textbf{Ostrosłup} & Podstawą jest wielokąt, a ściany
boczne są trójkątami spotykającymi się w jednym wierzchołku \\
\hline
\end{tabular}
\end{table}
\cite{cytat2}Figury przestrzenne, czyli mówiąc inaczej bryły geometryczne, to obiekty trójwymiarowe posiadające długość, szerokość i wysokość. Zajmują określoną objętość i mają powierzchnię składającą się z różnych ścian, krawędzi i wierzchołków.\\ \\
Do najczęściej spotykanych należą sześcian, prostopadłościan, walec, stożek, kula oraz ostrosłupy. Bryły obrotowe, jak kula czy walec, powstają przez obrót figur płaskich wokół osi.\\ \\
W matematyce oblicza się ich objętość i pole powierzchni, aby określić, ile miejsca zajmują oraz jak duża jest ich zewnętrzna powłoka. Bryły te opisują wiele przedmiotów w realnym świecie, dlatego są ważnym elementem geometrii.\\ \\


\section{Klasyfikacja i podstawowe własności figur przestrzennych}

\cite{cytat4}Figury przestrzenne można podzielić na kilka głównych grup, z których każda charakteryzuje się 
innymi właściwościami i budową. Najbardziej podstawowym podziałem jest rozróżnienie pomiędzy 
bryłami wypukłymi i wklęsłymi. Bryły wypukłe to takie, w których dowolne dwie punkty można 
połączyć odcinkiem w całości leżącym wewnątrz bryły. Bryły wklęsłe nie spełniają tej własności, 
co oznacza, że część takiego odcinka może znajdować się poza bryłą.\\

Drugim ważnym podziałem jest rozgraniczenie brył ze względu na kształt ich ścian. Bryły, których 
ściany są wielokątami, nazywamy wielościanami. Przykładem mogą być sześciany, graniastosłupy 
czy ostrosłupy. Z kolei bryły o ścianach krzywoliniowych, takie jak walce, stożki i kule, 
tworzą grupę brył obrotowych — ich powierzchnie powstają w wyniku obrotu pewnej figury płaskiej 
wokół wybranej osi.\\

Każda bryła przestrzenna posiada również charakterystyczne elementy: ściany, krawędzie oraz wierzchołki. 
Wielościany mają wszystkie te elementy dobrze zdefiniowane, natomiast bryły obrotowe mogą nie mieć 
krawędzi czy wierzchołków w ogóle, co stanowi jedną z zasadniczych różnic między tymi grupami brył. 
Takie rozróżnienie ułatwia analizę brył i pozwala uporządkować ich cechy geometryczne.

\chapter{Zastosowanie figur}


\cite{cytat1}Figury geometryczne są wszędzie wokół nas. W architekturze budynki
często opierają się na prostokątach i walcach, a w sztuce i
projektowaniu wykorzystuje się symetrię oraz proporcje geometryczne. W
informatyce geometria stanowi podstawę grafiki komputerowej, modelowania
3D i projektowania gier.\\
\cite{cytat3}Również w przyrodzie można zauważyć geometryczne wzory --- struktura
plastra miodu przypomina sześciokąty, a planety mają kształty zbliżone
do kul.\\



\textbf{Przykłady z życia}

\begin{enumerate}
\def\labelenumi{\arabic{enumi}.}
\item
  \textbf{Sześcian}

  \begin{enumerate}
  \def\labelenumii{\arabic{enumii}.}
  \item
    Kostka do gry
  \item
    pudełko w kształcie sześcianu
  \end{enumerate}
\item
  \textbf{Prostopadłościan}

  \begin{enumerate}
  \def\labelenumii{\arabic{enumii}.}
  \item
    Książka
  \item
    Cegła
  \end{enumerate}
\item
  \textbf{Kula}

  \begin{enumerate}
  \def\labelenumii{\arabic{enumii}.}
  \item
    Piłka
  \item
    Globus
  \end{enumerate}
\item
  \textbf{Ostrosłup}

  \begin{enumerate}
  \def\labelenumii{\arabic{enumii}.}
  \item
    Piramida egipska
  \item
    Stożek drogowy
  \end{enumerate}
\item
  \textbf{Stożek}

  \begin{enumerate}
  \def\labelenumii{\arabic{enumii}.}
  \item
    Rożek do lodów
  \item
    czapka urodzinowa
  \end{enumerate}

\end{enumerate}





\addcontentsline{toc}{chapter}{Bibliografia}
\markboth{Bibliografia}{Bibliografia}
\bibliographystyle{plain} 
\bibliography{bibliografia} 






\chapter{Wnioski}

W tym dokumencie wykorzystałem klasę \verb|report| zamiast \verb|book|.
Główne powody tego wyboru są następujące:

\begin{itemize}
    \item \textbf{Zakres pracy} – dokument ma kilka rozdziałów, ale nie jest pełnowymiarową książką. 
          Klasa \verb|report| jest przeznaczona właśnie dla raportów, prac zaliczeniowych i projektów.
    \item \textbf{Prostsza struktura} – \verb|report| nie wymaga części typu \textit{front matter}, 
          \textit{main matter}, \textit{back matter}, które są typowe dla \verb|book|.
    \item \textbf{Wystarczające możliwości} – klasa \verb|report| obsługuje rozdziały (\verb|\chapter|),
          sekcje i podsekcje, co w zupełności wystarcza do realizacji wymagań zadania.\\ \\
\end{itemize}


\textbf{Link do repozytorium:}
\url{https://github.com/ptest2901/Zadanie-Latex}



\end{document}
