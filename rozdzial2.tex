\chapter{Figury przestrzenne}

Figury przestrzenne (bryły) mają trzy wymiary: długość, szerokość i
wysokość. Ich objętość i pole powierzchni są kluczowe w wielu
zastosowaniach. Do najczęściej spotykanych należą:
\\

\begin{table}[H]
\begin{tabular}{| c | c | p{5cm} |}
\hline
\textbf{Numer} & \textbf{Figura} & \textbf{Opis} \\
\hline \hline
\textbf{1.} & \textbf{Sześcian} & wszystkie ściany są kwadratami, a
krawędzie mają równą długość. \\
\hline
\textbf{2.} & \textbf{Prostopadłościan} & ma sześć ścian
prostokątnych. \\
\hline
\textbf{3.} & \textbf{Kula} & zbiór wszystkich punktów w przestrzeni
równo oddalonych od środka. \\
\hline
\textbf{4.} & \textbf{Stożek i walec} & mają podstawy w kształcie koła,
różnią się jednak sposobem łączenia z wierzchołkiem lub drugą
podstawą. \\
\hline
\textbf{5.} & \textbf{Ostrosłup} & Podstawą jest wielokąt, a ściany
boczne są trójkątami spotykającymi się w jednym wierzchołku \\
\hline
\end{tabular}
\end{table}
\cite{cytat2}Figury przestrzenne, czyli mówiąc inaczej bryły geometryczne, to obiekty trójwymiarowe posiadające długość, szerokość i wysokość. Zajmują określoną objętość i mają powierzchnię składającą się z różnych ścian, krawędzi i wierzchołków.\\ \\
Do najczęściej spotykanych należą sześcian, prostopadłościan, walec, stożek, kula oraz ostrosłupy. Bryły obrotowe, jak kula czy walec, powstają przez obrót figur płaskich wokół osi.\\ \\
W matematyce oblicza się ich objętość i pole powierzchni, aby określić, ile miejsca zajmują oraz jak duża jest ich zewnętrzna powłoka. Bryły te opisują wiele przedmiotów w realnym świecie, dlatego są ważnym elementem geometrii.\\ \\


\section{Klasyfikacja i podstawowe własności figur przestrzennych}

\cite{cytat4}Figury przestrzenne można podzielić na kilka głównych grup, z których każda charakteryzuje się 
innymi właściwościami i budową. Najbardziej podstawowym podziałem jest rozróżnienie pomiędzy 
bryłami wypukłymi i wklęsłymi. Bryły wypukłe to takie, w których dowolne dwie punkty można 
połączyć odcinkiem w całości leżącym wewnątrz bryły. Bryły wklęsłe nie spełniają tej własności, 
co oznacza, że część takiego odcinka może znajdować się poza bryłą.\\

Drugim ważnym podziałem jest rozgraniczenie brył ze względu na kształt ich ścian. Bryły, których 
ściany są wielokątami, nazywamy wielościanami. Przykładem mogą być sześciany, graniastosłupy 
czy ostrosłupy. Z kolei bryły o ścianach krzywoliniowych, takie jak walce, stożki i kule, 
tworzą grupę brył obrotowych — ich powierzchnie powstają w wyniku obrotu pewnej figury płaskiej 
wokół wybranej osi.\\

Każda bryła przestrzenna posiada również charakterystyczne elementy: ściany, krawędzie oraz wierzchołki. 
Wielościany mają wszystkie te elementy dobrze zdefiniowane, natomiast bryły obrotowe mogą nie mieć 
krawędzi czy wierzchołków w ogóle, co stanowi jedną z zasadniczych różnic między tymi grupami brył. 
Takie rozróżnienie ułatwia analizę brył i pozwala uporządkować ich cechy geometryczne.
