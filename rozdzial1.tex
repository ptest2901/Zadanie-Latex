\chapter{Figury płaskie}

Figury płaskie to takie, mają tylko dwa wymiary -- długość i szerokość.\\

Do najważniejszych należą:

\begin{itemize}
\item
  \textbf{Trójkąt} -- figura o trzech bokach i trzech kątach; suma jego
  kątów zawsze wynosi 180°.
\item
  \textbf{Kwadrat} -- ma cztery równe boki i cztery kąty proste.
\item
  \textbf{Prostokąt} -- przeciwległe boki są równe, a wszystkie kąty
  proste.
\item
  \textbf{Koło} -- zbiór wszystkich punktów na płaszczyźnie w tej samej
  odległości od środka.\\
\end{itemize}

\includegraphics[scale=1]{Obraz1}\\

\textbf{Przykłady Figur płaskich}

\begin{enumerate}
\def\labelenumi{\arabic{enumi}.}
\item
  Kwadrat
\item
  Trójkąt
\item
  Koło\\
\end{enumerate}

\section{Podział figur płaskich}

\cite{cytat5}Figury płaskie można podzielić na dwie główne grupy: wielokąty oraz figury o krzywoliniowych 
brzegach. Wielokąty, takie jak trójkąty, czworokąty czy pięciokąty, są zbudowane z odcinków 
połączonych wierzchołkami. Figury krzywoliniowe, do których należy koło czy elipsa, mają brzegi 
złożone z linii zakrzywionych. Taki podział pozwala łatwiej porządkować figury i analizować ich 
właściwości.

\section{Elementy charakterystyczne figur płaskich}

Każda figura płaska posiada elementy, które pomagają ją opisać. W przypadku wielokątów są to boki, 
kąty oraz wierzchołki. Długości boków i miary kątów decydują o kształcie i typie figury. W figurach 
okrągłych głównymi elementami są promień, średnica oraz obwód. Dzięki tym elementom można określić 
wielkość figury, obliczyć jej pole oraz porównywać ją z innymi kształtami.





